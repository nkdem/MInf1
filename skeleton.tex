% UG project example file, February 2024
%
%   Added the "online" option for equal margins, February 2024 [Hiroshi Shimodaira, Iain Murray]
%   A minor change in citation, September 2023 [Hiroshi Shimodaira]
%
% Do not change the first two lines of code, except you may delete "logo," if causing problems.
% Understand any problems and seek approval before assuming it's ok to remove ugcheck.
\documentclass[logo,bsc,singlespacing,parskip,online]{infthesis}
\usepackage{ugcheck}


% Include any packages you need below, but don't include any that change the page
% layout or style of the dissertation. By including the ugcheck package above,
% you should catch most accidental changes of page layout though.

\usepackage{microtype} % recommended, but you can remove if it causes problems
\usepackage[square,numbers]{natbib} % recommended for citations
\bibliographystyle{plainnat}

\begin{document}
\begin{preliminary}

\title{Evaluation of Machine Learning Algorithms for Speech Prioritisation in Noisy Environments}

\author{Nikodem Bieniek}

%\course{Artificial Intelligence and Computer Science}
\course{Master of Informatics} % MInf students

\project{MInf Project (Part 1) Report}  % 4th year MInf students
%\project{MInf Project (Part 2) Report}  % 5th year MInf students


\date{\today}

\abstract{
This skeleton demonstrates how to use the \texttt{infthesis} style for
undergraduate dissertations in the School of Informatics. It also emphasises the
page limit, and that you must not deviate from the required style.
The file \texttt{skeleton.tex} generates this document and should be used as a
starting point for your thesis. Replace this abstract text with a concise
summary of your report.
}

\maketitle

\newenvironment{ethics}
   {\begin{frontenv}{Research Ethics Approval}{\LARGE}}
   {\end{frontenv}\newpage}

\begin{ethics}
\textbf{Instructions:} \emph{Agree with your supervisor which
statement you need to include. Then delete the statement that you are not using,
and the instructions in italics.\\
\textbf{Either complete and include this statement:}}\\ % DELETE THESE INSTRUCTIONS
%
% IF ETHICS APPROVAL WAS REQUIRED:
This project obtained approval from the Informatics Research Ethics committee.\\
Ethics application number: ???\\
Date when approval was obtained: YYYY-MM-DD\\
%
\emph{[If the project required human participants, edit as appropriate, otherwise delete:]}\\ % DELETE THIS LINE
The participants' information sheet and a consent form are included in the appendix.\\
%
% IF ETHICS APPROVAL WAS NOT REQUIRED:
\textbf{\emph{Or include this statement:}}\\ % DELETE THIS LINE
This project was planned in accordance with the Informatics Research
Ethics policy. It did not involve any aspects that required approval
from the Informatics Research Ethics committee.

\standarddeclaration
\end{ethics}


\begin{acknowledgements}
Any acknowledgements go here.
\end{acknowledgements}


\tableofcontents
\end{preliminary}


\chapter{Introduction}
\section{Motivations}
Hearing loss is a prevalent condition affecting as much as 
430 million people - or 1 in 18 people. This is expected 
to rise to 1 in 10 by 2050 \cite{WHO2024deafness}.
The most common treatment for hearing loss is the 
provision of hearing technology - such as Hearing Aid (HA) or cochlear implants.
There are many types of HAs, but the most common
type is the behind-the-ear (BTE) hearing aid \cite{Kochkin2010MarkeTrak8}.
However, HA users often report that they struggle to hear speech
in noisy environments. For example, in a study by Kochkin \cite{Kochkin2010MarkeTrak8},
42\% of HA users reported that wind noise was a significant issue for them.
This project aims to evaluate the effectiveness of machine learning algorithms
in prioritising the speech in various environments (such as windy environments).

Modern hearing aids now apply a wide range of techniques to achieve 
better speech prioritisation. For wind noise reduction, this can be achieved 
from mechanical solutions - product design to covers that reduce wind noise - to 
signal processing techniques to compensate for mechanical limitations.
However, current techniques are still not perfect as shown by the study from Kochkin.

Wind noise reduction and indeed, noise reduction in general, is a challenging problem
when paired with speech. This is because you have 
to strike a balance between reducing background noise and
preserving speech. 
Korhen's paper \cite{Korhonen2021WindNoise} outlines 
various techniques that could be used to reduce the wind noise in hearing aids -
from modulation-based noise reduction algorithms (Wiener filtering),
adaptive filtering algorithms, to machine learning techniques.
The paper mentions that the the proposed ML technique:
Long Short-Term Memory (LSTM) neural networks provided
modest improvements in wind noise reduction, however, it did highlight
that ML techniques may still have utility through further research 
and careful algorithmic choices. 

This project aims to pair the proposed machine learning techniques with
signal processing techniques to evaluate the effectiveness of 
speech prioritisation in noisy environments. The idea is to first 
perform acoustic scene analysis (ASA) to classify the environment
based on the audio signal. Afterwards, speech enhancement 
techniques will be applied to the signal to prioritise speech.
ASA can be done using a dataset of varying environments, and 
using machine learning techniques to classify the environment.

In this project, we will be using a novel dataset proposed by Huwel et al. \cite{Huwel2020HearDS}
This dataset (called HEAR-DS) is unique because it is specially tailored for HA signal processing and contains
various environments. Normally, voice activity detection (VAD) is used to
detect speech, however, the dataset helpfully contains 
labels which samples contain speech. This can be used 
to implicitly train the machine learning model to classify the environment 
and whether speech is present. Additionally, the paper presents an 
elementary example of how the dataset can be used: to classify the environment -
it showcases the use of a
convolutional neural network (CNN) to classify the environment.
This project will be extending the paper by actually using the dataset 
and comparing various machine learning techniques to evaluate the
effectiveness of the proposed machine learning techniques. 

This project will investigate how recurrent neural networks (RNN) and 
their variants, such as LSTMs, can be used to classify the environment.
 Based on the environment classified, the system will then apply
signal processing techniques to enhance the speech. 
The project 
however has to be mindful in its algorithmic choices - as the computational
power required in HA is limited. It is difficult to pinpoint the exact computational power of a HA
due to the proprietary nature of the devices. In August 2024,
Phonak (Sonova Holding AG) released a new HA which is their first AI 
equipped HA. The device is said to be capable of handling 7,700 Million 
Operations Per Second to accommmodate its neural network with 4.5 million parameters.  Contrast this with a paper 
from 2021 investigating techniques in VAD for hearing aids 
quotes that it `rarely exceeds 5 million instructions per second (MIPS)` \cite{Gomez2021MIPS}
Moreover, Apple's release of a FDA approved hearing aid in which was previously a mainstream earphone wearable,
Apple Airpods, also shows that there's more interest in this area.
So suffice to say, the computational power of hearing aids is accelerating 
and is most likely going to continue to grow. 

Delay constraints are also
important in HA, as the user needs to hear the speech in real-time.
For example, so long as the speech production is no higher 
than 30 milliseconds (ms), and ideally less than 20ms, the user
is unaffected by the delay \cite{Stone2002Delays}.


% Thus, the evaluation plays a crucial role in this project. From
% objective measures such as the short-time objective intelligibility (STOI)
% and perceptual evaluation of speech quality (PESQ), to quantitative
% measures such as multiply-accumulate operations (MACs) and delay requirements or
% real time factor (RTF) i.e. how fast a system can compute a speech result. 
% The combination of these measures will provide a comprehensive evaluation
% of the proposed machine learning techniques.
%

The project is predominantly aimed at the hearing aid industry. If successful, the project
could advance the state-of-the-art techniques in hearing aids.
Which in turn could improve the quality of life for hearing aid users.
A secondary goal is to make hearing aid research more accessible to the 
computer science community.
The project also hopes to benefit other fields that deal with audio signal processing,
such as mainstream wearables like headphones or microphones.

\chapter{Background}
% TODO: Add background information about the project
\chapter{Conclusions}

\section{Final Reminder}

The body of your dissertation, before the references and any appendices,
\emph{must} finish by page~40. The introduction, after preliminary material,
should have started on page~1.

You may not change the dissertation format (e.g., reduce the font size, change
the margins, or reduce the line spacing from the default single spacing). Be
careful if you copy-paste packages into your document preamble from elsewhere.
Some \LaTeX{} packages, such as \texttt{fullpage} or \texttt{savetrees}, change
the margins of your document. Do not include them!

Over-length or incorrectly-formatted dissertations will not be accepted and you
would have to modify your dissertation and resubmit. You cannot assume we will
check your submission before the final deadline and if it requires resubmission
after the deadline to conform to the page and style requirements you will be
subject to the usual late penalties based on your final submission time.

\bibliography{mybibfile}


% You may delete everything from \appendix up to \end{document} if you don't need it.
\appendix

\chapter{First appendix}

\section{First section}

Any appendices, including any required ethics information, should be included
after the references.

Markers do not have to consider appendices. Make sure that your contributions
are made clear in the main body of the dissertation (within the page limit).

\chapter{Participants' information sheet}

If you had human participants, include key information that they were given in
an appendix, and point to it from the ethics declaration.

\chapter{Participants' consent form}

If you had human participants, include information about how consent was
gathered in an appendix, and point to it from the ethics declaration.
This information is often a copy of a consent form.


\end{document}
